\section{The general Construction}
The tangent bundle to an $n$-manifold $M$ is a $2n$-manifold called
$TM$ naturally constructed in terms of $M$. As a set, $TM$ is the
disjoint union of the tangent spaces $T_p M$. We will now describe 
the construction in detail.

\begin{definition}
Let $(U,\varphi)$ and $(V,\psi)$ be charts around $p \in M$.
Let $u \in T_{\varphi(p)}\varphi(u)$ and $v\in T_{\psi(p)}\psi(v)$.
Then $(U,\varphi, u)$ and $(V, \psi, v)$ are called equivalent when
\[
D(\psi \circ \varphi^{-1})(\varphi(p))(u) = v
\]
This is an equivalence relation, utilizing the chain rule.

The set of equivalence classes of such triples is called the 
\textbf{tangent space} to $p$ of $M$, denoted $T_p M$.
\end{definition}

$T_p(M)$ is a real vector space of dimension $\dim M$ and $D(\psi \circ \phi^{-1})$ is 
a linear isomorphism.

As a set, the tangent bundle $TM$ is
\[ 
TM=\bigsqcup_{p \in M} T_p M 
\]
equipped with a natural projection $\pi : TM \to M$.
