\section{Smooth Maps}
If $M$ and $N$ are topological manifolds, the natural notion of a 
morphism is $f : M \to N$ continuous. Then two topological manifolds
$M$, $N$, are equivalent if there exists a homeomorphism $f : M \to N$. 
Topological manifolds together with continuous maps form a category.

\begin{definition}
A \textbf{category} consists of objects $C$ and arrows $A$. Each arrow
goes from some object $C$ (the source) to another object (the target).
These arrows are often called \textbf{morphisms}. Furthermore, for each
object $c \in C$ there exists a morphism $1_c \in A$ such that the source
and target of $1_c$ are both $c$. In addition, morphisms can be composed
and composition satisfies additivity. For $x, y \in C$ we write 
$\Hom(x,y) \subset A$ as the set of morphisms from $x$ to $y$. Stated this
way, composition becomes
\[ \circ : \Hom(x,y) \times \Hom(y,z) \to \Hom(x,z) \]
\[ \circ : (f,g) \mapsto g \circ f \]
\end{definition}

\begin{example}
The category of topological manifolds is defined by $C$ being the class of
all topological manifolds with arrows $A$ being continuous maps between them.
\end{example}

For smooth manifolds, we need the correct notion of a morphism.

\begin{definition}
A map $f : M \to N$, $M$, $N$ smooth manifolds, is called \textbf{smooth}
when for each chart $(U,\varphi)$ for $M$ and each chart $(V,\psi)$ for 
$N$, the composition
\[
\psi \circ f \circ \varphi^{-1} \in C^\infty(\varphi(U),\R^n).
\]
The set of smooth maps from $M$ to $N$ is denoted $C^\infty (M,N)$. A smooth
map with smooth inverse is called a \textbf{diffeomorphism}.
\end{definition}

\begin{lemma}
If $g : L \to M$ and $f : M \to N$ are smooth maps, then so is 
$f \circ g : L \to N$.
\end{lemma}
